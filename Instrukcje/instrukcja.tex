\documentclass{article}
\usepackage[utf8]{inputenc}

\usepackage[T1]{fontenc}
\usepackage[polish]{babel}
\title{Superfarmer}
\author{Opracowanie zbiorowe}



\begin{document}

\maketitle

\section{Opis gry}
Superfarmer jest grą o hodowaniu zwierząt przeznaczoną dla 4 graczy. W tej edycji gracz będzie mierzył się z 3 botami w wyścigu o tytuł Superfarmera. Zwycięzcą zostaje ten, kto jako pierwszy zgromadzi w swoim stadzie co najmniej po jednym ze zwierząt: konia, krowę, świnę, owcę i królika. Należy jednak uważać na różne zagrożenia.


\section{Przebieg rozgrywki}
Każdy z graczy po kolei wykonuje swój ruch. Każda kolejka składa się z dwóch części: wymiany i rozmnażania zwierząt.
\subsection{Wymiana}
Przed rzutem kośćmi gracz może podjąć się wymiany zwierząt ze swojego stada z stadem głównym po ustalonych kursach wymiany.
\\
wstawić tu jakąś grafikę
\\
Wolno graczowi zakupić jedno zwierzę lub jedno zwierzę sprzedać.
Wolno również gracz, w tej wersji gry, dokonać wymiany po kursie gorszym dla gracza. (np. kupić konia za 4 krowy, lub sprzedać konia za 1 krowę)
\subsection{Rozmnażanie}
Gracz rzuca dwiema kośćmi. Następnie otrzymuje z głównego stada tyle zwierząt z gatunków, które pojawiły się na kościach, ile pełnych par tych gatunków posiada. Jeśli w głównym stadzie brakuje zwierząt to gracz bezpowrotnie traci do nich prawo.
\subsection{Utrata zwierząt}
Jeśli na kości wypadł lis, to gracz traci małego psa (jeśli go posiada) lub wszystkie swoje króliki (w przeciwnnym przypadku).
Jeśli na kości wypadł wilk, to gracz traci dużego psa (jeśli go posiada) lub wszystkie swoje zwierzęta poza królikami (w przeciwnym przypadku).




\subsection{Częste pytania}

\end{document}

